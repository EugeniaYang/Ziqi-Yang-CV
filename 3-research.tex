
%-----------RESEARCH EXPERIENCE-----------------
\section{\textbf{Research Experience}}
  \resumeSubHeadingListStart
  \resumeSubheading {Graduate Student Research Assistant, Northeastern University}{Boston, MA}{Advisor: Professor Dakuo Wang}{May. 2023 - Present}
    \vspace{-1.0mm}
      \resumeItemListStart
        % \item{\textbf{Talk2Care study (submitted paper):} (1) Designed and conducted semi-structured interviews with older adults(N=10) and healthcare providers(N=9) to study challenges during \textbf{patient-provider communication} for \textbf{older adults at home}, and opportunities for LLM-powered human-AI systems.} \\
        % (2) Implemented a pilot \textbf{LLM-powered system, Talk2Care},  with an \textbf{voice assistant} (Alexa Echo) for older adults and a \textbf{dashboard interface} (Figma and React) for healthcare providers. \\
        % (3) Conducted 2 user studies to reveal good system usability, potentials and design implications for LLM-powered systems to facilitate and mediate patient-provider communication.
        % \item{\textbf{In-depth qualitative study in cancer care: } Exploring communication challenges for gastrointestinal cancer patients and post-surgery scenarios; Designing and conducting interview studies with patients, providers and caregivers.}
        % \item{\textbf{System deployment and field study: } Designing and developing a LLM-powered system for GI cancer providers to follow up with patients' post-surgery recovery at home for around 30 and later around 200 participants}

         \item \textbf{Interviewed} 19 participants to uncover communication challenges during patient-provider communication for older adults at home, and opportunities for \textbf{Large-Language-Model(LLM)}-powered systems.
        \item Designed and implemented an \textbf{LLM-powered system}, Talk2Care,  with an \textbf{voice assistant (Alexa Echo)} for older adults and a \textbf{dashboard interface (Figma and React)} for healthcare providers to facilitate asynchronous patient-provider communication.
        \item Conducted \textbf{user studies for two groups} to reveal good system usability, discussed potentials and design implications for LLM-powered systems as a patient-provider communication facilitator and mediator. Conducting in-depth interviews and a 6-week real-world deployment.
        % \item First-author paper submitted to CHI'24 (under review). See \href{https://arxiv.org/abs/2309.09357}{here}.
        % \item Studying challenges for gastrointestinal(GI) cancer care and post-surgery scenarios; developing an LLM-powered system for large-scale deployment study. 
      \resumeItemListEnd


\resumeSubheading {\href{https://drive.google.com/file/d/1R8OSR61UOyUseOvf2t_uAYz-JDCtbojh/view?usp=drive_link}{Graduate Student Research Assistant, UM}}{Ann Arbor, MI}{Advisor: Professor Mark Newman, Professor Predrag Klasnja}{Oct. 2022 - Present}
\vspace{-1mm}
      \resumeItemListStart
% \item{\textbf{Pre-study stage:} Designed 2*2 scripted \textbf{coach chatbots} with different\textbf{ empathy and personality} traits to investigate their roles in building long-term user-chatbot relationships in health interventions; conducted 6 prototype-based pilot studies and interviews. Analyzed user studies results on Amazon Mechanical Turk in \textbf{questionnaire} ratings and \textbf{interview} transcripts using statistical analysis and open coding, summarized preferences and limitations}
%       \item{\textbf{Exploratory study stage:} Planning an Experiment-in-a-Box(XB) study using technology probes to explore scenarios and interactions that users build trust with AI chatbots, revised study instructions and interview questions}
%       \item{\textbf{LLM-powered chatbot study:} Designing and developing an chatbot for JITAI\footnote{Just-In-Time-Adaptive Intervention} to promote physical health.}


        \item Designed \textbf{scripts and Figma prototypes} for four coach chatbots with empathy and personality traits to investigate their roles in building long-term user-chatbot relationships in health interventions.
        \item Conducted six pilot \textbf{user studies}; analyzed \textbf{questionnaire} ratings and \textbf{interview} transcripts for user studies to reveal user preferences towards informal empathetic chatbots, and personalized needs.
      \item Planning an exploratory study using technology probes to seek scenarios and interactions that users build trust with AI chatbots; developing an \textbf{AI chatbot} for Just-In-Time-Adaptive Intervention to promote physical health.
      \resumeItemListEnd


% \resumeSubheading {Student Researcher}{Shanghai, China}{\textbf{Instructor:} Dr. Aalap Doshi, UM; Master's Capstone and Research Project}{Aug 2022 - Dec 2022}
%         \vspace{-2.0mm}
%       \resumeItemListStart
%       \item{Conducted literature review and case studies on inclusive design for older adults.}
%       \item{Conducted surveys (N=30) and user interviews (N=5) on older adults'}
%       \item{Designed ; tested prototype }
%       \resumeItemListEnd

%     \vspace{-3.0mm}



% \resumeSubheading
%       {Student Researcher}{Ann Arbor, MI}
%       {\textbf{Instructor:} Professor Robin Brewer (SI552 Intro to Accessibility)}{Oct. 2022 - Dec. 2022}
%             \vspace{-2.0mm}
%       \resumeItemListStart
%       \item Facebook for People with Dementia - Critique and Redesign \href{https://drive.google.com/file/d/1w6TjnP9pJaZ9fD8Ijr2u9arrXOBgAb_C/view?usp=drive_link}{\textbf{[Project Paper]}}
%         % \item{Conducted literature review and comparative analysis on the \textbf{accessibility} of features in contemporary \textbf{social network sites} (SNS) such as Facebook for \textbf{older adults with dementia}.}
%         \item{Critiqued and redesigned Facebook features to improve \textbf{accessibility} for \textbf{older adults with dementia} by providing context, enhancing social connections, and protecting privacy and autonomy.}
%       \resumeItemListEnd
%       \vspace{-3.0mm}
%     \resumeSubheading
%       {Student Researcher}
% {Ann Arbor, 2022}{\textbf{Instructor:} Professor Nazanin Andalibi (SI529 Online Communities)}{Jan. 2022 - Mar. 2022}
% \vspace{-2.0mm}
%       \resumeItemListStart
%       \item Bilibili/Cake-Making: An Online Community Analysis \href{https://drive.google.com/file/d/1yKaZgQ0l5ZG7jOVQh-FoweCtiBrTROZ9/view?usp=drive_link}{\textbf{[Project Paper]}}
%         \item{Proposed research plan for video content-creating platform Bilibili Cake-Making channel.}
%         \item{Conducted observation to uncover trends, interaction variation, and social etiquette in bullet screens creator-creator, and creator-audience interactions; discussed design implications to promote user engagement.}
%       \resumeItemListEnd
%     \vspace{-3.0mm}

\resumeSubheading{Multidisciplinary Design Programme(MDP), UM }{Ann Arbor, MI}{Advisor: Professor Amy Chavasse}{Jan. 2022 - Dec. 2022}
\vspace{-1.0mm}
\resumeItemListStart
\item Conducted desktop research on The Shape and Flow of Languages in choreography, arts, and technology.
\item Sketched 360 $ ^\circ $ design and made \textbf{physical prototype}, design and implemented an interactive \href{https://hubs.mozilla.com/HUwkA6r/sturdy-experienced-exploration}{\textbf{VR exhibition in Mozilla Hubs}} and constructed \href{https://yangziqi0605.wixsite.com/shape-flow-language}{\textbf{portfolio website using Wix}}.
\resumeItemListEnd
    \vspace{-1.0mm}

\resumeSubheading{UX Researcher for Underground Printing, UM}{Ann Arbor, MI}{Instructor: Professor Joyojeet Pal}{Aug. 2021 - Dec. 2021}
\vspace{-1.0mm}
\resumeItemListStart
\item Conducted \textbf{survey} and \textbf{five interviews} with three major user groups of online apparel company Underground Printing; using \textbf{affinity diagrams} to analyze user needs, pain points, and constructed persona. 
\item Conducted \textbf{usability testing} and \textbf{heuristic evaluation} for mobile design and ordering experience, and provided design suggestions and implications.
\resumeItemListEnd
\vspace{-2.0mm}

\resumeSubheading {\href{https://drive.google.com/file/d/1qVbvJj6CCOpjq4kctIF9TNRu89aw3mPW/view?usp=sharing}{Undergraduate Student Research Assistant}, SJTU}{Shanghai, China}{Advisor: Professor Peisen Huang}{May. 2021 - Aug. 2021}
        \vspace{-1.0mm}
      \resumeItemListStart
        % \item{Conducted literature review on camera calibration, computer vision technology and applications for camera calibration and robotic movement in VR pose estimation tasks.}
        % \item{Designed integration and control solutions for high-precision camera and AGV robotic arm. Tested AGV robotic arm with Python scripts for practical pose-estimation tasks.}
        \item{Designed integration and control solutions for a \textbf{high-precision camera} on \textbf{Automated Guided Vehicles (AGV)} and robotic arm; tested control solutions with Python scripts.}
      \resumeItemListEnd

  \resumeSubHeadingListEnd

 \vspace{-8.0mm}
% %-----------RESEARCH EXPERIENCE-----------------
% \section{\textbf{Research Experience}}
%   \resumeSubHeadingListStart
%   \resumeSubheading {Graduate Student Research Assistant}{Boston, MA}{\textbf{Advisor}: Professor Dakuo Wang, Northeastern University}{May. 2023 - Present}
%     \vspace{-2.0mm}
%       \resumeItemListStart
%         \item{\textbf{Talk2Care study (submitted paper):} (1) Designed and conducted semi-structured interviews with older adults(N=10) and healthcare providers(N=9) to study challenges during \textbf{patient-provider communication} for \textbf{older adults at home}, and opportunities for LLM-powered human-AI systems.} \\
%         (2) Implemented a pilot \textbf{LLM-powered system, Talk2Care},  with an \textbf{voice assistant} (Alexa Echo) for older adults and a \textbf{dashboard interface} (Figma and React) for healthcare providers. \\
%         (3) Conducted 2 user studies to reveal good system usability, potentials and design implications for LLM-powered systems to facilitate and mediate patient-provider communication.
%         \item{\textbf{In-depth qualitative study in cancer care: } Exploring communication challenges for gastrointestinal (GI) cancer patients and post-surgery scenarios; Designing and conducting interview studies with patients, providers and caregivers.}
%         \item{\textbf{System deployment and field study: } Designing and developing a LLM-powered system for GI cancer providers to follow up with patients' post-surgery recovery at home for around 30 and later around 200 participants}
%       \resumeItemListEnd

%     \vspace{-3.0mm}

        

% % \resumeSubheading {Student Researcher}{Shanghai, China}{\textbf{Instructor:} Dr. Aalap Doshi, UM; Master's Capstone and Research Project}{Aug 2022 - Dec 2022}
% %         \vspace{-2.0mm}
% %       \resumeItemListStart
% %       \item{Conducted literature review and case studies on inclusive design for older adults.}
% %       \item{Conducted surveys (N=30) and user interviews (N=5) on older adults'}
% %       \item{Designed ; tested prototype }
% %       \resumeItemListEnd

% %     \vspace{-3.0mm}



% \resumeSubheading
%       {Student Researcher}{Oct. 2022 - Dec. 2022}
%       {\textbf{Instructor:} Professor Robin Brewer; Course Project in SI552 Intro to Accessibility}{}
%             \vspace{-2.0mm}
%       \resumeItemListStart
%       \item \textbf{Project: Facebook for People with Dementia - Critique and Redesign \href{https://drive.google.com/file/d/1w6TjnP9pJaZ9fD8Ijr2u9arrXOBgAb_C/view?usp=drive_link}{\textbf{[Project Paper]}}}
%         \item{Conducted literature review and comparative analysis on the \textbf{accessibility} of features in contemporary \textbf{social network sites} (SNS) such as Facebook for \textbf{older adults with dementia}.}
%         \item{Critiqued and redesigned Facebook features to improve accessibility for users with dementia by providing context, enhancing social connections, and protecting privacy and autonomy.}
%       \resumeItemListEnd
%       \vspace{-3.0mm}
%     \resumeSubheading
%       {Student Researcher}
% {Ann Arbor, 2022}{\textbf{Instructor:} Professor Nazanin Andalibi; Course Project in SI529 Online Communities}{Jan. 2022 - Mar. 2022}
% \vspace{-2.0mm}
%       \resumeItemListStart
%       \item \textbf{Project: Bilibili/Cake-Making: An Online Community Analysis \href{https://drive.google.com/file/d/1yKaZgQ0l5ZG7jOVQh-FoweCtiBrTROZ9/view?usp=drive_link}{\textbf{[Project Paper]}}}
%         \item{Proposed research plan for video content-creating platform Bilibili Cake-Making channel based on relevant literature.}
%         \item{Conducted community observation to uncover trends, interaction variation, and social etiquette in bullet screens creator-creator, and creator-audience interactions; discussed design implications to promote community engagement.}
%       \resumeItemListEnd
%     \vspace{-3.0mm}


% \resumeSubheading {Undergraduate Student Research Assistant}{Shanghai, China}{\textbf{Advisor}: Professor Peisen Huang, SJTU}{May. 2021 - Aug. 2021}
%         \vspace{-2.0mm}
%       \resumeItemListStart
%       \item{\textbf{Project: High-Precision Pose Estimation and Camera  Calibration Based on Computer Vision \href{https://drive.google.com/file/d/1qVbvJj6CCOpjq4kctIF9TNRu89aw3mPW/view?usp=sharing}{\textbf{[Report]}} }}
%         \item{Conducted literature review on camera calibration, computer vision technology and applications for camera calibration and robotic movement in VR pose estimation tasks.}
%         \item{Designed integration and control solutions for high-precision camera and AGV robotic arm. Tested AGV robotic arm with Python scripts for practical pose-estimation tasks.}
%       \resumeItemListEnd

%     \vspace{-3.0mm}
%   \resumeSubHeadingListEnd
% \vspace{-5.0mm}



    


% \resumeSubheading{Food Insecurity in the United States - Data Analysis and Visualization}{}{Student Data Analyst}{}
% \vspace{-2.0mm}
% \resumeItemListStart
% \item Based on datasets published by U.S. government departments, use python, altair, Tableau to analyze factors influencing food insecurity in the U.S., write reports and suggestions based on outcomes
% \resumeItemListEnd
        
    % \vspace{-3.0mm}

% \resumeSubheading{Project Leader, Designer, Software Developer, SJTU}{Shanghai, China}{\textbf{Advisor:} Professor Mian Li, Professor Zhanxun Dong, SJTU}{Oct. 2020 - Oct. 2021}
% \resumeSubheading{Intelligent Illustration System for SJTU Info Security and IoT\footnote{Internet of Things} Based on Big Data}{Shanghai, China}{Project Leader, Designer, Software Developer}{Oct. 2020 - Oct. 2021}
% \vspace{-2mm}
% \resumeItemListStart
% \item Designed and developed the website based on Vue and React framework with FastApi and MySql backend to process and illustrate statistical information. Recruited and managed team members in collaborative work.
% \resumeItemListEnd
    % \vspace{-3.0mm}
% \resumeSubheading{“The Magic Wand” as in Harry Potter Series Using AI Technologies}{}
% {Student Designer, Developer}{}
% \vspace{-2.0mm}
% \resumeItemListStart
% \item Built an interactive device that recognize multimodal user input and make sound/light effects like a magic wand proposed in novels by applying AI algorithms in Python to Arduino board, Raspberry Pi and relevant tool kit
% \resumeItemListEnd